\documentclass[12pt]{article}
\usepackage[a4paper, margin=1in]{geometry}
\usepackage{titlesec}
\usepackage{hyperref}
\usepackage{longtable}

\titleformat{\section}{\large\bfseries}{\thesection}{1em}{}
\titleformat{\subsection}{\normalsize\bfseries}{\thesubsection}{1em}{}
\titleformat{\subsubsection}{\normalsize\itshape}{\thesubsubsection}{1em}{}

\title{Software Requirements Specification (SRS) \\ \vspace{1em} \Large Intelligent Cyber Threat Intelligence System}
\author{Group: Security Insights \\ Version: 4.0}
\date{\today}

\begin{document}

\maketitle
\thispagestyle{empty}

\newpage
\tableofcontents
\newpage

\section*{Revision History}
\begin{longtable}{|p{2cm}|p{3cm}|p{3cm}|p{5cm}|}
\hline
\textbf{Version} & \textbf{Date} & \textbf{Author(s)} & \textbf{Description} \\
\hline
1.0 & 2024-11-28 & Security Insights & Initial draft. Monolithic design. \\
\hline
2.0 & 2024-11-30 & Security Insights & Microservices architecture and multi-database integration. \\
\hline
3.0 & 2024-12-01 & Security Insights & Enhanced Azure deployment, CTI-specific pipelines, and dashboards. \\
\hline
3.1 & 2024-12-02 & Security Insights & Improved RBAC, integration with SIEM/SOAR, and compliance-focused revisions. \\
\hline
4.0 & \today & Security Insights & Comprehensive improvements, including risk analysis, testing traceability, and scalability enhancements. \\
\hline
\end{longtable}

\newpage

\section{Introduction}
\subsection{Purpose}
This document specifies requirements for developing a robust Cyber Threat Intelligence (CTI) system for a Fortune 10 company. The system will:
\begin{itemize}
    \item Integrate structured, unstructured, and relationship-based threat data across relational, document, and graph databases.
    \item Provide real-time threat ingestion, enrichment, and scoring pipelines.
    \item Enable scalability and resilience via microservices on Azure Kubernetes Service (AKS).
    \item Offer actionable insights to C-suite executives and SOC analysts via interactive dashboards.
\end{itemize}

\subsection{Scope}
The application addresses the needs of enterprise-level CTI by:
\begin{itemize}
    \item Automating ingestion from threat feeds (OSINT, commercial, proprietary).
    \item Supporting real-time processing and enrichment pipelines.
    \item Aligning with frameworks such as MITRE ATT&CK.
    \item Integrating with SIEM/SOAR platforms for actionable intelligence.
    \item Ensuring security compliance with GDPR and ISO 27001.
\end{itemize}

\subsection{Document Overview}
The SRS is structured into the following sections:
\begin{enumerate}
    \item Introduction: Goals, scope, and definitions.
    \item Overall Description: High-level product perspective and operating environment.
    \item Specific Requirements: Detailed functional and non-functional requirements.
    \item System Architecture: Design details for deployment and microservices.
    \item Security: Authentication, authorization, and data protection measures.
    \item Appendices: Supporting diagrams and documentation.
\end{enumerate}

\subsection{Definitions, Acronyms, and Abbreviations}
\begin{itemize}
    \item CTI: Cyber Threat Intelligence
    \item SIEM: Security Information and Event Management
    \item SOAR: Security Orchestration, Automation, and Response
    \item RBAC: Role-Based Access Control
    \item AKS: Azure Kubernetes Service
\end{itemize}

\subsection{References}
\begin{enumerate}
    \item \href{https://spring.io/projects/spring-boot}{Spring Boot Documentation}
    \item \href{https://azure.microsoft.com/}{Azure Services Documentation}
    \item \href{https://attack.mitre.org/}{MITRE ATT&CK Framework}
    \item \href{https://owasp.org}{OWASP Security Standards}
\end{enumerate}

\section{Overall Description}
\subsection{Product Perspective}
This system integrates backend APIs and CTI pipelines into a scalable, modular architecture:
\begin{itemize}
    \item **Databases**: MySQL for structured data, MongoDB for nested datasets, Neo4j for relationship traversal.
    \item **Deployment**: Hosted on Azure AKS with global redundancy.
    \item **Integration**: Real-time SIEM/SOAR integration for enriched threat data dissemination.
\end{itemize}

\subsection{Product Features}
\begin{itemize}
    \item Ingests and enriches threat data from multiple sources.
    \item Supports multi-database CRUD operations via REST APIs.
    \item Provides role-based dashboards with KPIs and trend analysis.
    \item Ensures compliance with regulatory standards.
\end{itemize}

\subsection{Assumptions and Dependencies}
\begin{itemize}
    \item Azure cloud services will remain operational and available globally.
    \item Threat feed providers will deliver data in agreed formats.
    \item End-users will have high-speed internet connections.
\end{itemize}

\subsection{Operating Environment}
\begin{itemize}
    \item Backend: Java Spring Boot.
    \item Databases: MySQL, MongoDB, Neo4j.
    \item Deployment: Azure Kubernetes Service (AKS).
    \item Analytics: Azure Machine Learning and Power BI.
\end{itemize}

\section{Specific Requirements}
\subsection{Functional Requirements}
\begin{itemize}
    \item FR-1: Provide REST APIs for CRUD operations on all databases.
    \item FR-2: Implement real-time ingestion pipelines via Azure Event Hubs.
    \item FR-3: Support role-based access for data retrieval and processing.
    \item FR-4: Enrich threats using MITRE ATT&CK mapping.
    \item FR-5: Deliver data to SIEM/SOAR systems in enriched format.
\end{itemize}

\subsection{Non-functional Requirements}
\begin{itemize}
    \item NFR-1: Ensure 99.99\% uptime with automated failover.
    \item NFR-2: Handle 50k+ IOCs/day with scalable microservices.
    \item NFR-3: API response time must not exceed 200ms under load.
    \item NFR-4: Encrypt all sensitive data at rest and in transit.
\end{itemize}

\section{System Architecture}
\subsection{Microservices Design}
Each microservice is independent, focusing on a specific database or task:
\begin{itemize}
    \item Relational DB Microservice: `/mysql/api/v1/...`
    \item Document DB Microservice: `/mongodb/api/v1/...`
    \item Graph DB Microservice: `/neo4j/api/v1/...`
    \item Enrichment Service: Integrates MITRE ATT&CK and predictive analytics.
\end{itemize}

\subsection{Azure Deployment Architecture}
\begin{itemize}
    \item AKS for containerized deployment.
    \item Azure SQL Database for relational storage.
    \item Azure Cosmos DB for document storage.
    \item Azure Monitor and Sentinel for observability.
\end{itemize}

\section{Risk Assessment}
\subsection{Risk Table}
\begin{longtable}{|p{3cm}|p{4cm}|p{4cm}|p{4cm}|}
\hline
\textbf{Risk} & \textbf{Impact} & \textbf{Likelihood} & \textbf>Mitigation> \\ 
\hline
Service Outage & High & Moderate & Implement AKS redundancy and failover systems. \\ 
\hline
Data Breach & Critical & Low & Encrypt data and use Azure Key Vault. \\ 
\hline
API Rate Limits & Medium & High & Throttle API requests and implement caching. \\ 
\hline
\end{longtable}

\section{Testing Traceability}
\subsection{Traceability Matrix}
\begin{longtable}{|p{3cm}|p{4cm}|p{5cm}|}
\hline
\textbf{Requirement} & \textbf{Test Case ID} & \textbf{Testing Tool} \\
\hline
FR-1 & TC-01 & Postman API tests \\
\hline
FR-2 & TC-02 & JMeter load test \\
\hline
\end{longtable}

\end{document}
