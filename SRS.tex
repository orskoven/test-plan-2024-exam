\documentclass[12pt]{article}
\usepackage[a4paper, margin=1in]{geometry}
\usepackage{titlesec}
\usepackage{hyperref}
\usepackage{longtable}

\titleformat{\section}{\large\bfseries}{\thesection}{1em}{}
\titleformat{\subsection}{\normalsize\bfseries}{\thesubsection}{1em}{}
\titleformat{\subsubsection}{\normalsize\itshape}{\thesubsubsection}{1em}{}

\title{Software Requirements Specification (SRS) \\ \vspace{1em} \Large Intelligent Cyber Threat Intelligence System}
\author{Group: Security Insights \\ Version: 3.1}
\date{\today}

\begin{document}

\maketitle
\thispagestyle{empty}

\newpage
\tableofcontents
\newpage

\section*{Revision History}
\begin{longtable}{|p{2cm}|p{3cm}|p{3cm}|p{5cm}|}
\hline
\textbf{Version} & \textbf{Date} & \textbf{Author(s)} & \textbf{Description} \\
\hline
1.0 & 2024-11-28 & Security Insights & Initial draft. Monolithic design. \\
\hline
2.0 & 2024-11-30 & Security Insights & Microservices architecture and multi-database integration. \\
\hline
3.0 & 2024-12-01 & Security Insights & Enhanced Azure deployment, CTI-specific pipelines, and dashboards. \\
\hline
3.1 & \today & Security Insights & Improved RBAC, integration with SIEM/SOAR, and compliance-focused revisions. \\
\hline
\end{longtable}

\newpage

\section{Introduction}
\subsection{Purpose}
This document specifies requirements for developing a robust Cyber Threat Intelligence (CTI) system for a Fortune 10 company. The system will:
\begin{itemize}
    \item Integrate structured, unstructured, and relationship-based threat data across relational, document, and graph databases.
    \item Provide real-time threat ingestion, enrichment, and scoring pipelines.
    \item Enable scalability and resilience via microservices on Azure Kubernetes Service (AKS).
    \item Offer actionable insights to C-suite executives and SOC analysts via interactive dashboards.
\end{itemize}

\subsection{Scope}
The application addresses the needs of enterprise-level CTI by:
\begin{itemize}
    \item Automating ingestion from threat feeds (OSINT, commercial, proprietary).
    \item Supporting real-time processing and enrichment pipelines.
    \item Aligning with frameworks such as MITRE ATT&CK.
    \item Integrating with SIEM/SOAR platforms for actionable intelligence.
\end{itemize}

\subsection{Definitions, Acronyms, and Abbreviations}
\begin{itemize}
    \item CTI: Cyber Threat Intelligence
    \item SIEM: Security Information and Event Management
    \item SOAR: Security Orchestration, Automation, and Response
    \item RBAC: Role-Based Access Control
    \item AKS: Azure Kubernetes Service
\end{itemize}

\subsection{References}
\begin{enumerate}
    \item \href{https://spring.io/projects/spring-boot}{Spring Boot Documentation}
    \item \href{https://azure.microsoft.com/}{Azure Services Documentation}
    \item \href{https://attack.mitre.org/}{MITRE ATT&CK Framework}
    \item OWASP Security Standards
\end{enumerate}

\section{Overall Description}
\subsection{Product Perspective}
This system integrates backend APIs and CTI pipelines into a scalable, modular architecture:
\begin{itemize}
    \item **Databases**: MySQL for structured data, MongoDB for nested datasets, Neo4j for relationship traversal.
    \item **Deployment**: Hosted on Azure AKS with global redundancy.
    \item **Integration**: Real-time SIEM/SOAR integration for enriched threat data dissemination.
\end{itemize}

\subsection{Product Features}
\begin{itemize}
    \item Ingests and enriches threat data from multiple sources.
    \item Supports multi-database CRUD operations via REST APIs.
    \item Provides role-based dashboards with KPIs and trend analysis.
\end{itemize}

\subsection{User Classes and Characteristics}
\begin{itemize}
    \item **SOC Analysts**: Detailed access to enriched threat feeds.
    \item **Threat Hunters**: Advanced query capabilities (e.g., graph traversals).
    \item **Executives**: High-level dashboards and trend summaries.
    \item **Auditors**: Read-only access for compliance.
\end{itemize}

\subsection{Operating Environment}
\begin{itemize}
    \item Backend: Java Spring Boot.
    \item Databases: MySQL, MongoDB, Neo4j.
    \item Deployment: Azure Kubernetes Service (AKS).
    \item Analytics: Azure Machine Learning and Power BI.
\end{itemize}

\section{Specific Requirements}
\subsection{Functional Requirements}
\begin{itemize}
    \item FR-1: Provide REST APIs for CRUD operations on all databases.
    \item FR-2: Implement real-time ingestion pipelines via Azure Event Hubs.
    \item FR-3: Support role-based access for data retrieval and processing.
    \item FR-4: Enrich threats using MITRE ATT&CK mapping.
    \item FR-5: Deliver data to SIEM/SOAR systems in enriched format.
\end{itemize}

\subsection{Non-functional Requirements}
\begin{itemize}
    \item NFR-1: Ensure 99.99\% uptime with automated failover.
    \item NFR-2: Handle 50k+ IOCs/day with scalable microservices.
    \item NFR-3: API response time must not exceed 200ms under load.
\end{itemize}

\subsection{Database Design Requirements}
\begin{itemize}
    \item MySQL: Store structured CTI data (campaigns, vulnerabilities).
    \item MongoDB: Store unstructured feed data in denormalized JSON format.
    \item Neo4j: Enable relationship queries for actor-TTP-campaign mapping.
\end{itemize}

\section{System Architecture}
\subsection{Microservices Design}
The system utilizes microservices for each database and enrichment task:
\begin{itemize}
    \item Relational DB Microservice: `/mysql/api/v1/...`
    \item Document DB Microservice: `/mongodb/api/v1/...`
    \item Graph DB Microservice: `/neo4j/api/v1/...`
\end{itemize}

\subsection{Azure Deployment Architecture}
\begin{itemize}
    \item AKS for containerized deployment.
    \item Azure SQL Database for relational storage.
    \item Azure Cosmos DB for document storage.
    \item Azure Monitor and Sentinel for security and observability.
\end{itemize}

\section{Security and RBAC}
\subsection{Authentication and Authorization}
\begin{itemize}
    \item OAuth2 and JWT-based session management.
    \item Fine-grained RBAC for data and API endpoints.
    \item Audit trails for all role actions and API requests.
\end{itemize}

\subsection{Data Protection}
\begin{itemize}
    \item Data encryption at rest and in transit.
    \item Azure Key Vault for secure credentials storage.
\end{itemize}

\section{Data Science Pipeline Integration}
\subsection{Features}
\begin{itemize}
    \item Real-time anomaly detection in threat feeds.
    \item Predictive analytics for attack forecasting.
    \item TTP clustering and malware classification using Azure ML.
\end{itemize}

\subsection{Frameworks}
\begin{itemize}
    \item Python for data processing.
    \item Azure Machine Learning for model training and deployment.
\end{itemize}

\section{Executive Dashboard}
\subsection{Features}
\begin{itemize}
    \item Key Metrics: Threat volumes, average detection-to-mitigation time.
    \item Visualizations: Heatmaps, actor campaign timelines, and attack vectors.
    \item Data Export: PDF or CSV summaries for board presentations.
\end{itemize}

\section{Testing and Compliance}
\subsection{Testing Scenarios}
\begin{itemize}
    \item Stress test ingestion pipelines for 50k+ IOCs/day.
    \item Red team simulation for adversarial activity detection.
    \item Database failover and recovery tests.
\end{itemize}

\subsection{Compliance}
\begin{itemize}
    \item GDPR: Pseudonymize IP and user logs.
    \item ISO 27001: Define ISMS for secure system operations.
\end{itemize}

\section{Appendices}
\begin{itemize}
    \item Appendix A: ER Diagrams for MySQL.
    \item Appendix B: Swagger API Documentation.
    \item Appendix C: Kubernetes Deployment YAML.
    \item Appendix D: Sample Threat Dashboard Mockup.
\end{itemize}

\end{document}
