\documentclass[12pt]{article}
\usepackage[a4paper, margin=1in]{geometry}
\usepackage{titlesec}
\usepackage{hyperref}
\usepackage{longtable}

\titleformat{\section}{\large\bfseries}{\thesection}{1em}{}
\titleformat{\subsection}{\normalsize\bfseries}{\thesubsection}{1em}{}
\titleformat{\subsubsection}{\normalsize\itshape}{\thesubsubsection}{1em}{}

\title{Software Requirements Specification (SRS) \\ \vspace{1em} \Large Intelligent Cyber Threat Intelligence System}
\author{Group: Security Insights \\ Version: 4.1}
\date{\today}

\begin{document}

\maketitle
\thispagestyle{empty}

\newpage
\tableofcontents
\newpage

\section*{Revision History}
\begin{longtable}{|p{2cm}|p{3cm}|p{3cm}|p{5cm}|}
\hline
\textbf{Version} & \textbf{Date} & \textbf{Author(s)} & \textbf{Description} \\
\hline
1.0 & 2024-11-28 & Security Insights & Initial draft. Monolithic design. \\
\hline
2.0 & 2024-11-30 & Security Insights & Microservices architecture and multi-database integration. \\
\hline
3.0 & 2024-12-01 & Security Insights & Enhanced Azure deployment, CTI-specific pipelines, and dashboards. \\
\hline
3.1 & 2024-12-02 & Security Insights & Improved RBAC, integration with SIEM/SOAR, and compliance-focused revisions. \\
\hline
4.0 & 2024-12-05 & Security Insights & Comprehensive improvements for traceability, scalability, and testing strategies. \\
\hline
4.1 & \today & Security Insights & Focused enhancements addressing testing, risk mitigation, and practical exam preparation. \\
\hline
\end{longtable}

\newpage

\section{Introduction}
\subsection{Purpose}
This document specifies the requirements for developing a robust Cyber Threat Intelligence (CTI) system tailored for enterprise-grade threat detection and response. Key objectives include:
\begin{itemize}
    \item Real-time ingestion and analysis of threat data across diverse formats and sources.
    \item Integration with SIEM/SOAR systems for actionable intelligence.
    \item Role-specific dashboards catering to SOC analysts, threat hunters, executives, and auditors.
\end{itemize}

\subsection{Scope}
The system focuses on enhancing enterprise cybersecurity by:
\begin{itemize}
    \item Automating ingestion and enrichment of structured, unstructured, and relational threat intelligence data.
    \item Providing predictive analytics using data science pipelines for proactive threat mitigation.
    \item Adhering to industry standards such as GDPR, ISO 27001, and MITRE ATT&CK.
\end{itemize}

\subsection{Document Overview}
The SRS is structured as follows:
\begin{enumerate}
    \item **Introduction**: Project goals, definitions, and references.
    \item **Overall Description**: High-level view of the system’s architecture and features.
    \item **Specific Requirements**: Functional and non-functional requirements.
    \item **Testing and Risk Management**: Testing strategies, traceability, and risk mitigation plans.
    \item **System Architecture**: Details of microservices, deployment, and database design.
    \item **Appendices**: Supporting diagrams, sample API documentation, and YAML configurations.
\end{enumerate}

\newpage

\section{Overall Description}
\subsection{Product Features}
\begin{itemize}
    \item **Real-time Threat Ingestion**: Automate the ingestion of 50k+ Indicators of Compromise (IOCs) daily.
    \item **Data Enrichment**: Use MITRE ATT&CK mapping to classify tactics, techniques, and procedures (TTPs).
    \item **Actionable Dashboards**: Role-specific insights for operational, tactical, and strategic decision-making.
    \item **Predictive Analytics**: Detect patterns and predict threats using machine learning models.
    \item **Compliance Monitoring**: Audit trails, GDPR-compliant pseudonymization, and ISO 27001 ISMS integration.
\end{itemize}

\subsection{Preconditions}
\begin{itemize}
    \item Threat feeds are accessible via authenticated APIs.
    \item The Azure cloud environment supports scalable deployments.
    \item End-users are trained to interpret predictive analytics and dashboard KPIs.
\end{itemize}

\subsection{Assumptions and Dependencies}
\begin{itemize}
    \item External APIs will maintain uptime and data quality.
    \item The organization uses modern browsers to access dashboards.
    \item No major disruptions in Azure cloud services.
\end{itemize}

\newpage

\section{Specific Requirements}
\subsection{Functional Requirements}
\begin{itemize}
    \item **FR-1**: Provide APIs for CRUD operations on relational, document, and graph databases.
    \item **FR-2**: Ingest threat feeds from OSINT, commercial, and proprietary sources via Azure Event Hubs.
    \item **FR-3**: Enrich threat data with MITRE ATT&CK mapping and predictive analytics.
    \item **FR-4**: Provide role-based access control (RBAC) for different user groups.
    \item **FR-5**: Export enriched data to SIEM/SOAR platforms in JSON format.
\end{itemize}

\subsection{Non-functional Requirements}
\begin{itemize}
    \item **NFR-1**: Ensure 99.99\% system uptime with automated failover mechanisms.
    \item **NFR-2**: Support a minimum of 50k IOCs/day with 200ms API response time under load.
    \item **NFR-3**: Ensure data encryption at rest (AES-256) and in transit (TLS 1.3).
\end{itemize}

\subsection{Testing Traceability Matrix}
\begin{longtable}{|p{3cm}|p{4cm}|p{5cm}|}
\hline
\textbf{Requirement} & \textbf{Test Case ID} & \textbf{Testing Tool/Methodology} \\
\hline
FR-1 & TC-001 & API tests using Postman \\
\hline
FR-2 & TC-002 & JMeter stress testing for ingestion pipelines \\
\hline
FR-3 & TC-003 & Unit testing with Mockito for enrichment logic \\
\hline
FR-4 & TC-004 & RBAC testing using Selenium for UI workflows \\
\hline
FR-5 & TC-005 & API contract tests with Swagger Validator \\
\hline
\end{longtable}

\newpage

\section{Testing and Risk Management}
\subsection{Testing Strategies}
\begin{itemize}
    \item **Unit Tests**: Validate individual modules (e.g., data enrichment services) using the AAA pattern.
    \item **Integration Tests**: Test API endpoints with mock external dependencies.
    \item **End-to-End Tests**: Use Selenium to automate dashboard workflows for SOC analysts and executives.
    \item **Performance Tests**: Conduct load, stress, and spike testing with JMeter to simulate ingestion of 50k IOCs/day.
\end{itemize}

\subsection{Risk Assessment}
\begin{longtable}{|p{3cm}|p{4cm}|p{4cm}|p{4cm}|}
\hline
\textbf{Risk} & \textbf{Impact} & \textbf{Likelihood} & \textbf{Mitigation} \\ 
\hline
Data Breach & Critical & Medium & Encrypt data and use Azure Key Vault for credentials. \\ 
\hline
API Downtime & High & High & Implement retries and caching for API calls. \\ 
\hline
Integration Errors & Medium & Low & Use CI pipelines with automated integration tests. \\ 
\hline
\end{longtable}

\newpage

\section{System Architecture}
\subsection{Microservices Design}
\begin{itemize}
    \item **Database Services**: Separate microservices for relational (MySQL), document (MongoDB), and graph (Neo4j) databases.
    \item **Enrichment Services**: Python-based services integrating MITRE ATT&CK and Azure Machine Learning.
    \item **Dashboard Services**: Frontend services deployed on AKS, backed by Power BI for visualization.
\end{itemize}

\subsection{Azure Deployment}
\begin{itemize}
    \item **Redundancy**: Multi-region deployment for global failover.
    \item **Monitoring**: Azure Monitor and Sentinel for observability.
    \item **Security**: RBAC and Key Vault for sensitive configuration data.
\end{itemize}

\section{Appendices}
\begin{itemize}
    \item **Appendix A**: ER Diagram for MySQL.
    \item **Appendix B**: Sample API Contract (Swagger documentation).
    \item **Appendix C**: YAML Configuration for Kubernetes Deployment.
    \item **Appendix D**: Mockups of SOC and Executive Dashboards.
\end{itemize}

\end{document}
